%\pagestyle{empty}
%\linespread{1.2}
%\documentclass[11pt,letterpaper,headinclude=true]{scrbook}
%\usepackage[linedheaders, dottedtoc, manychapters,
%subfigure,floatperchapter]{classicthesis}
%\textheight 22.8 cm
\documentclass[phd, titlesmallcaps, copyrightpage]{mqthesis}

%FORMATO USUAL DE LA ESCUELA DESCOMENTAR PARA USAR
%\documentclass[11pt,letterpaper]{report}
% \oddsidemargin 4.6 mm
% \evensidemargin 4.6 mm
%\textwidth 16.6 cm
%\topmargin -5.4 mm
%\textheight 22.8 cm
% \headsep 0.5cm
% \parindent 1em
\parskip 1.8ex
%\renewcommand{\topfraction}{0.85}
%\renewcommand{\textfraction}{0.1}
%\renewcommand{\floatpagefraction}{0.75}
%\documentclass[10 pt ,letterpaper, twosided, twocolumn]{report}

\newcommand\nop[1]{}
%\usepackage{graphicx}
\usepackage{natbib}
\usepackage{amsmath}
\usepackage[activeacute,english]{babel}
\usepackage[ansinew]{inputenc}
\usepackage{latexsym}
\usepackage[varg]{txfonts}
\usepackage{subfigure}
\usepackage{longtable,lscape}
\usepackage{threeparttable}
\usepackage{verbatim}
\usepackage{rotating}
\usepackage{afterpage} 
\usepackage{amssymb}
%\usepackage[format=hang]{caption,subfig}
\usepackage{color, colortbl}

\newcommand{\comments}[1]{}
\newcommand{\tco}{\mbox{$^{13}$CO(3-2)}}
\newcommand\nhp{\mbox{N$_2$H$^+$}}
\newcommand\nhpt{N$_2$H$^+$(3-2)}
\newcommand\nhpl{\mbox{N$_2$H$^+$(1-0)}}
\newcommand\hcop{HCO$^+$}
\newcommand\hcn{HCN}
\newcommand\htrcn{H$^{13}$CN}
\newcommand\hnc{HNC}
\newcommand\hnco{HNCO}
\newcommand\ch{C$_2$H}
\newcommand\htcop{H$^{13}$CO$^+$}
\newcommand\hctn{HC$_3$CN}
\newcommand\hntc{HN$^{13}$C}
\newcommand\tcs{$^{13}$CS}
\newcommand\chtcn{CH$_3$CN}
\newcommand\sio{SiO}
\newcommand\halpha{H$_\alpha$}
\newcommand\hctrccn{HC$^{13}$CCN}
\newcommand\mslin{M$_\odot$ pc$^{-1}$}
\newcommand\ms{M$_\odot$}
\newcommand\kms{ km s$^{-1}$}
\newcommand\kcm{ K cm$^{-3}$}
\newcommand\grcm{gr cm$^{-1}$}
\newcommand\grcmt{gr cm$^{-3}$}
\newcommand\cmd{cm$^{-2}$}

\newcommand\cmdgr{cm$^2$gr$^{-1}$}
\newcommand\meanp{$\langle P \rangle$}
\newcommand\meand{$\langle \rho \rangle$}
\newcommand\meansigma{$\langle \sigma \rangle$}
\newcommand\ratiop{$\langle P \rangle/P_s$}
\newcommand\ratiom{$m/m_{vir}$}
\newcommand\mw{$\mathcal{M}/|\mathcal{W}|$}
\newcommand{\mum}{$\mu$m}
\newcommand{\degr}{\ensuremath{^\circ}}




%\renewcommand{\baselinestretch}{.4}
%\renewcommand{\tco}{$^{13}$CO(3-2)}
%\renewcommand{\nhp}{N$_2$H$^+$}
%%% Para pasar del dvi al ps usar en Windows::::  dvips -t letter cosa
%%% para transformar bien a ps usar dvips -t letter cosa.dvi
%\usepackage{psfig}
%\usepackage{dvips}
%\usepackage{psfrag,xscale}

%\bibpunct{[**}{**]}{,}{}{}{,}

\numberwithin{equation}{chapter}
%\usepackage{makeidx}
%\numberwithin{figure}{chapter}

\setcounter{secnumdepth}{3}% para numerar hasta subsubsections
\setcounter{tocdepth}{2} % hasta donde aparece en el indice


%\numberwithin{equation}{subsection}
%\usepackage{astron}
%\usepackage[spanish]{babel} % T�tulos en espa�ol

% Cambiar el titulo de la bibliografia cuando el estilo es report
%\renewcommand\bibname{Bibliography}%Bibliography es el dafault
%\renewcommand\bibname{Bibliography\footnote{Gracias ADS!!!}}%Bibliography es el dafault

% \renewcommand\refname{Bibliography} % Asi es cuando es article

%\input{aas_macros.tex}
\bibliographystyle{astron}
\bibpunct{(}{)}{;}{a}{}{,}
\def\newblock{}
\def\apj{{ApJ}}                 
\def\apjl{{ApJ}}                
\def\apjs{{ApJS}}               
\def\ao{{Appl.~Opt.}}           
\def\apss{{Ap\&SS}}             
\def\aap{{A\&A}}                
\def\aapr{{A\&A~Rev.}}          
\def\aaps{{A\&AS}}              
\def\azh{{AZh}}                 
\def\baas{{BAAS}}               
\def\jrasc{{JRASC}}             
\def\memras{{MmRAS}}            
\def\mnras{{MNRAS}}   
\def\qjras{{QJRAS}}   
\def\araa{{ARAA}}         
\def\pra{{Phys.~Rev.~A}}        
\def\prb{{Phys.~Rev.~B}}        
\def\prc{{Phys.~Rev.~C}}        
\def\prd{{Phys.~Rev.~D}}        
\def\pre{{Phys.~Rev.~E}}        
\def\prl{{Phys.~Rev.~Lett.}}    
\def\pasp{{PASP}}               
\def\pasj{{PASJ}}               
\def\planss{{Planet.~Space~Sci.}}


%\bibliographystyle{plainnat}
\DeclareMathOperator{\F}{F}
\DeclareMathOperator{\BesselK}{K}
\DeclareMathOperator{\sech}{sech}
\DeclareMathOperator{\n}{N}
\DeclareMathOperator{\B}{B}

\makeatletter
\renewcommand\paragraph{\@startsection{paragraph}{4}{\z@}%
  {-3.25ex\@plus -1ex \@minus -.2ex}%
  {1.5ex \@plus .2ex}%
  {\normalfont\normalsize\bfseries}}
\makeatother
%\makeindex
\begin{document}

\vspace*{-18 ex}
\thispagestyle{empty}
\begin{minipage}[b]{0.1\linewidth}
\includegraphics[width=.9 \textwidth ]{ImageServlet.ps}%Figures/logo.ps}
\end{minipage}
\begin{minipage}[b]{0.9\linewidth}
{\bf \large \noindent UNIVERSIDAD DE CHILE \\
   FACULTAD DE CIENCIAS F\'ISICAS Y MATEM\'ATICAS \\
 DEPARTAMENTO  DE ASTRONOM\'IA }
\end{minipage}
\hfill
%\vspace*{-5 ex}\includegraphics[width= 0.1 \textwidth ]{Figures/logo.ps}
%\vspace{2 ex} %para el formato normal
\vspace{4cm}

\begin{center}{\bf \Large TITULO AQUI}\end{center}

\vspace{1 ex} 

\begin{center}{\it  \large TESIS PARA OPTAR AL GRADO DE DOCTORADO EN \\
CIENCIAS, MENCI\'ON ASTRONOM\'IA} \end{center}

\vspace{1.5cm} 


\begin{center} {\large AUTOR: \\ {\bf ELISE MARIE GERMAINE SERVAJEAN BERGOEING}} \end{center}

\vspace{7 ex} 

\begin{center} { PROFESOR GU\'IA:\\
    GUIDO GARAY BRIGNARDELLO\\
    PROFESOR CO-GU\'IA:\\
    JILL MAREE RATHBORNE\\
    \vspace{0.3cm}
    MIEMBROS DE LA COMISI\'ON:\\
    ??\\
    ??\\
    ??\\
    \vspace{0.8cm}
    SANTIAGO DE CHILE\\
    ?? 2015 } \end{center}




\newpage

\thispagestyle{empty}
{\Huge \bf \vspace*{5 ex} Resumen}

\vspace*{8 ex}%\hspace*{3em} 
\input{resumen.tex}
\newpage

\thispagestyle{empty}
{\Huge \bf \vspace*{5 ex} Agradecimientos}
\vspace*{8 ex}%\hspace*{3em}
%\input{agradecimientos.tex}
\newpage

\pagenumbering{Roman}
\tableofcontents
\listoffigures
\listoftables
%{\tiny
%%%%%%%%%%%%%%%%%%%%%%%%%%%%%%%%%%%%%%%%%%%%%%%%%%%%%%%%%%%%%%%%%%%%%%%%%%%%%%%%%%%%%%%%%%%%%%%%%%%%%%%%%%%%%%%%%%%%%%%%%%%

\chapter{Introduction}
\pagenumbering{arabic}
\input{introduction/molecular_clouds}
\input{introduction/star_formation}
\include{introduction/filaments}



 
%\section{Physical introduction}


\newpage
\section{Identification and characterization of filaments}
\input{techniques/overview.tex}
\input{techniques/defining_filaments.tex}
\input{techniques/dust-emission.tex}
\input{techniques/millimeter.tex}

%\input{techniques/properties.tex}
%\input{techniques/distance.tex}
%\input{techniques/temperature.tex}
\label{sec:mass}
\input{techniques/pressure.tex}
\input{techniques/clumps.tex}

\newpage
\section{Thesis aims}
\input{introduction/thesis_aim.tex}


%%%%%%%%%%%%%%%%%%%%%%%%%%%%%%%%%%%%%%%%%%%%%%%%%%%%%%%%%%%%%%%%%%%%%%%%%%%%%%%%%%%%%%%%%%%%%%%%%%%%
\chapter{The data}\label{observations}
%\input{observaciones/intro_observations}
\input{observaciones/chap_observations}

\chapter{Galactic plane continuum survey: ATLASGAL}\label{sec:atlasgal}
\input{iden_filaments/atlasgal.tex}

\input{iden_filaments/overview.tex}
%%\input{chap_identification}


%%%%%%%%%%%%%%%%%%%%%%%%%%%%%%%%%%%%%%%%%%%%%%%%%%%%%%%%%%%%%%%%%%%%%%%%%%%%%%%%%%%%%%%%%%%%%%%%%%%%
%\chapter{Data description}\label{analy-obs}
\chapter{Filament A: Nessie}
\input{A/chap_analysis_obs}
\include{A/properties}
\chapter{Filament B: AGAL337.406-0.402}
\input{B/chap_analysis_obs_b}
\include{B/properties}
\chapter{Filament C: AGAL335.061-0.427}
\input{C/chap_analysis_obs_c}
\include{C/properties_c}
\chapter{Filament D: AGAL332.294-0.094}
\input{D/chap_analysis_obs_d}
\include{D/properties_d}
\chapter{Filament E: AGAL332.094-0.421}
\input{E/chap_analysis_obs_e}
\include{E/properties_e}
%\chapter{Physical parameters}\label{Model}
%\input{chap_parameters}


%%%%%%%%%%%%%%%%%%%%%%%%%%%%%%%%%%%%%%%%%%%%%%%%%%%%%%%%%%%%%%%%%%%%%%%%%%%%%%%%%%%%%%%%%%%%%%%
\chapter{Discussion \& Conclusions}
\input{resultados/global-prop.tex}
\input{resultados/stability.tex}
\input{resultados/fragmentation.tex}
\input{resultados/clumps.tex}
%\input{resultados/theories.tex}
%\input{resultados/evol_spectra.tex}
\input{resultados/tabla-sf-mol.tex}
%%%%%%%%%%%%%%%%%%%%%%%%%%%%%%%%%%%%%%%%%%%%%%%%%%%%%%%%%%%%%%%%%%%%%%%%%%%%%%%%%%%%%%%%%%%%%%%%%%%%%%%%%%%%%
\chapter{Summary}
\input{summary.tex}



% \renewcommand{\theenumi}{{\bf Ad.\arabic{enumi}}}
% \begin{enumerate}
% \item
% \end{enumerate}
% \renewcommand{\theenumi}{{\bf Co.\arabic{enumi}}}
% \begin{enumerate}
% \item
% \label{conplow}}
% \end{enumerate} 
% \renewcommand{\theenumi}{\arabic{enumi}}
\newpage
%%%%%%%%%%%%%%%%%%%%%%%%%%%%%%%%%%%%%%%%%%%%%%%%%%%%%%%%%%%%%%%%%%%%%%%%%%%%%%%%%%%%%%%%%%%%%%%%%%%%%%%
%%%%%%%%%%%%%%%%%%%%%%%%%%%%%%%%%%%%%%%%%%%%%%%%%%%%%%%%%%%%%%%%%%%%%%%%%%%%%%%%%%%%%%%%%%%%%%%%%%%%%%%%

\appendix %\addcontentsline{toc}{chapter}{Appendices}
\input{appendix/apendix.tex}
\label{apendix:parametros}
\input{appendix/apendix-param.tex}
\input{appendix/apendix-catalog.tex} 
\chapter[Equations]{}
\input{appendix/apendix-formulas.tex}
\input{appendix/channel_maps.tex}
%\input{resumenmass}
%\end{appendices}
%\input{apend_cal}


\newpage
%\noocite{libritoverde}
%\nocite{rybicky}
%\nocite{Shuradiation}
%\addcontentsline{toc}{chapter}{Bibliography}% 
\bibliography{bibliografia}      % bibliografia.bib
%}
\end{document}
