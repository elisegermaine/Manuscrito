%describe APEX observations\\
%- mention Spitzer, Herschel (refer to Yanett and Andres' papers on this)\\
%- SuperMALT refer yanett's 2014a paper\\

\begin{document}

\section{Source selection}
We observed four massive and dense cold cores (MDCC) discovered in a dust continuum emission survey towards luminous IRAS sources (\cite{2004ApJ...610..313G}). This MDCC were identified by having a large flux density at millimeter wavelengths, implying that they are massive, and by being undetected at mid-infrared (MSX) and far-infrared (IRAS) wavelengths, implying that they are cold. This suggests that our four MDCC are close to the initial conditions expected for massive star formation. The characteristics of the cores are shown in Table \ref{targets}.

\begin{table}[!ht]
\centering
\footnotesize
\caption{Characteristics of the observed cores.Note: this table is from Guido's Paper, change it with our info.}
\label{targets}
\begin{tabular}{l c c  c c  c c}
\\
\hline
SIMBA Source & \multicolumn{2}{c}{Peak Position} & \multicolumn{2}{c}{1.2 mm} & \multicolumn{2}{c}{CS (2 $\rightarrow$ 1)} \\
\hline \\
& $\alpha$ (J2000.0) & $\delta$ (J2000.0) & Flux & $\Theta_{\rm s}$ & $\Delta V$ & $\int~ T_{\rm mb}~ dv$ \\
& & & (Jy) & (arcsec) & ($km~ s^{-1}$) & ($K~ km~ s^{-1}$) \\
\hline \\
G305.136+0.068 & 13 10 41.7 & -62 43 15.5 & 4.24 & 33 & 4.15 $\pm$ 0.04 & 4.62 $\pm$ 0.04 \\
G333.125-0.562 & 16 21 34.9 & -50 41 10.2 & 8.88 & 40 & 3.90 $\pm$ 0.03 & 3.32 $\pm$ 0.03 \\
G18.606-0.076  & 18 25 08.7 & -12 45 26.9 & 1.17 & 23 &   . . . & . . . \\
G34.458+0.121  & 18 53 19.8 & +01 28 21.8 & 2.55 & 26 & 3.38 $\pm$ 0.03 & 3.30 $\pm$ 0.03 \\
\hline \\
%Note.—Units of right ascension are hours, minutes, and seconds, and units of declination are degrees, arcminutes,
%and arcseconds.
%a FWHM angular size.
\end{tabular}
\end{table}

\section{Molecular Lines}

We made molecular line observations using the 12m Atacama Pathfinder Experiment (APEX) located in Llano de Chajnantor, Chile. This observations were carried out during May and July 2012 with the SHeFI instrument. A detailed description of the characteristics of APEX is given by \cite{APEX}. The frontend consisted of a single pixel heterodyne SiS operating in the 345 GHz band. The backend consisted ... The half-power beam width of the telescope at 345 GHz is $\sim$20 arcsec. The main beam efficiency is 0.73. We mapped the N$_2$H$^$(3-2) and HCN(3-2) emission within a region of 100'x100'' and the HCO$^+$(3-2) emission within a region of 120''x120'', in all cases with angular spacings of 20'' and centered at the peak of the dust core.  On source integration time per map per position is shown on Table 1 (this info it's not here yet). \\

We also used the publicly avalaible data from \textit{The Millimetre Astronomy Legacy Team 90 GHz} (MALT90) Survey. A detailed description of the characteristics of MALT90 is given by \cite{MALT90} (agregar referencia). The observations were made with the Mopra Telescope, a 22m single-dish radio telescope located at ..., Australia.


\section{Dust Continuum}

Herschel observations?


\end{document}